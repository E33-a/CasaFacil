\section{Resumen del Sistema}
	\noindent El sistema \textit{Casa Fácil} incluye una serie de funcionalidades esenciales que permiten a los usuarios interactuar eficientemente entre sí y con el sistema. A continuación, se detallan todas las funcionalidades clave:
	
	\subsection*{Autenticación y Gestión de Usuarios}
		\begin{itemize}
			\item \textbf{Registro de usuarios}: Permite a nuevos usuarios registrarse mediante un formulario en línea o mediante OAuth.
			\item \textbf{Inicio de sesión}: Los usuarios pueden iniciar sesión con sus credenciales.
			\item \textbf{Recuperación de contraseña}: Los usuarios pueden recuperar el acceso a su cuenta mediante un enlace enviado a su correo electrónico.
			\item \textbf{Gestión de roles y permisos}: El sistema permite diferenciar funcionalidades según el rol (usuario autenticado, usuario no autenticado, propietario o administrador).
			\item \textbf{Verificación de identidad del propietario}: El sistema debe verificar la identidad del propietario como dueño o administrador del inmueble.
			\item \textbf{Cierre de sesión seguro}: El sistema permite a los usuarios cerrar su sesión de forma segura.
			\item \textbf{Gestión de perfil}: Los usuarios pueden actualizar su información personal, foto y preferencias desde su perfil.
			\item \textbf{Eliminación o desactivación de cuenta}: Los usuarios pueden eliminar o desactivar su cuenta conforme a la política de privacidad.
		\end{itemize}
	
	\subsection*{Gestión de Propiedades}
		\begin{itemize}
			\item \textbf{Visualización de propiedades}: Los usuarios pueden ver las propiedades disponibles en la plataforma con información básica.
			\item \textbf{Detalles de propiedad}: Al seleccionar una propiedad, se muestran su información adicional, incluyendo galería, descripción, precio, ubicación, subcategorías y reseñas.
			\item \textbf{Creación de propiedades}: Los usuarios con rol propietario pueden registrar nuevas propiedades para rentar.
			\item \textbf{Edición y eliminación de propiedades}: Los propietarios o administradores pueden modificar o eliminar propiedades previamente publicadas.
			\item \textbf{Publicación programada de propiedades}: Los propietarios pueden seleccionar una fecha futura para la publicación automática de sus propiedades.
			\item \textbf{Revisión/moderación de propiedades por administrador}: Las propiedades registradas son revisadas por un administrador antes de ser visibles en la plataforma.
		\end{itemize}
	
	\subsection*{Búsqueda y Exploración}
		\begin{itemize}
			\item \textbf{Búsqueda de propiedades}: El sistema permite a los usuarios buscar propiedades mediante filtros avanzados (precio, tipo, ubicación, categoría).
			\item \textbf{Búsqueda en mapa interactivo}: Los resultados de búsqueda se visualizan geográficamente sobre un mapa interactivo.
			\item \textbf{Búsqueda por cercanía a servicios clave}: El sistema permite filtrar propiedades según su proximidad a lugares de interés (universidades, transporte, hospitales, etc.).
		\end{itemize}
	
	\subsection*{Interacción del Usuario}
		\begin{itemize}
			\item \textbf{Favoritos}: Los usuarios pueden marcar propiedades como favoritas para acceder fácilmente a ellas.
			\item \textbf{Solicitud de reserva}: Los usuarios autenticados pueden enviar una solicitud para reservar una propiedad.
			\item \textbf{Agendamiento de citas}: Los usuarios pueden agendar visitas presenciales o virtuales con los propietarios.
			\item \textbf{Reserva de propiedad}: Los usuarios pueden reservar temporalmente una propiedad mediante pago anticipado.
			\item \textbf{Mensajería interna}: Los usuarios pueden enviar mensajes a los propietarios directamente desde la plataforma.
			\item \textbf{Calificación y reseña}: Los usuarios pueden calificar y dejar reseñas después de completar una transacción.
			\item \textbf{Reporte de usuarios o publicaciones}: Los usuarios pueden reportar publicaciones sospechosas o usuarios con comportamiento inadecuado.
			\item \textbf{Notificaciones internas y por correo}: El sistema envía notificaciones relevantes dentro de la aplicación y por medio de correo electrónico.
		\end{itemize}
	
	\subsection*{Pagos, Transacciones y Facturas}
		\begin{itemize}
			\item \textbf{Opciones de pago}: El sistema permite a los usuarios realizar pagos mediante la pasarela de pago de Mercado Pago.
			\item \textbf{Generación de comprobante}: Tras una transacción exitosa, se genera un comprobante en PDF con los detalles.
			\item \textbf{Historial de rentas}: Los usuarios pueden consultar su historial de rentas.
			\item \textbf{Gestión de reembolsos}: Los usuarios pueden solicitar reembolsos según las políticas establecidas.
		\end{itemize}
	
	\subsection*{Panel de Administración}
		\begin{itemize}
			\item \textbf{Panel de administración}: El sistema permite a los administradores gestionar usuarios, propiedades, reservas y transacciones.
			\item \textbf{Gestión de reportes de usuarios}: Los administradores pueden revisar reportes realizados por los usuarios sobre contenidos o comportamientos inadecuados.
			\item \textbf{Estadísticas por periodo de tiempo}: El sistema permite visualizar métricas clave como usuarios activos, ingresos y publicaciones.
			\item \textbf{Exportación de datos y reportes}: Los administradores pueden exportar reportes en formatos como CSV, PDF, Excel, etc.
		\end{itemize}
	
	\subsection*{Servicios Complementarios}
		\begin{itemize}
			\item \textbf{Integración de servicios locales}: El sistema conecta con APIs de mapas y servicios externos para mostrar información sobre lavanderías, estacionamientos, transporte y otros servicios cercanos a las propiedades.
		\end{itemize}
	
	\subsection*{Seguridad y Control de Acceso}
		\begin{itemize}
			\item \textbf{Control de roles y permisos}: El sistema permite diferenciar funcionalidades según el rol (usuario autenticado, usuario no autenticado, propietario o administrador).
			\item \textbf{Verificación de identidad del propietario}: El sistema debe verificar la identidad del propietario como dueño o administrador del inmueble.
			\item \textbf{Autenticación y autorización}: El sistema gestiona el acceso mediante JWT/OAuth y controla qué funcionalidades puede usar cada usuario según su rol.
		\end{itemize}
	
	\subsection*{Arquitectura y Diseño Técnico}
		\begin{itemize}
			\item \textbf{Microservicios}: El sistema se divide en múltiples servicios pequeños, autónomos y especializados, como el servicio de cuentas, el servicio de pago, el servicio de mapas, y el servicio de mensajería.
			\item \textbf{API Gateway}: Centraliza las solicitudes del cliente y las enruta a los servicios correspondientes.
			\item \textbf{Base de datos}: Almacena datos relacionales (PostgreSQL) y no estructurados (MongoDB), junto con caché y sesiones (Redis).
			\item \textbf{Event Sourcing}: Almacena eventos en lugar de estados actuales, facilitando la trazabilidad y auditoría.
			\item \textbf{Service Registry \& Discovery}: Facilita la comunicación entre microservicios sin necesidad de direcciones IP fijas.
			\item \textbf{Backend for Frontend (BFF)}: Crea capas intermedias específicas para cada tipo de cliente (web, móvil), adaptando las respuestas del backend a sus necesidades.
		\end{itemize}
	
	\subsection*{Diseño y Experiencia de Usuario}
		\begin{itemize}
			\item \textbf{Interfaz intuitiva}: La interfaz debe ser fácil de entender y usar, con flujos de usuario claros y retroalimentación inmediata.
			\item \textbf{Accesibilidad}: El sistema debe cumplir con estándares de accesibilidad para usuarios con discapacidad.
			\item \textbf{Responsividad}: La interfaz debe ajustarse automáticamente a diferentes tamaños de pantalla y resoluciones.
			\item \textbf{Localización e internacionalización}: Soporte para múltiples idiomas y formatos regionales.
		\end{itemize}
		
	\subsection*{Funcionalidades Específicas}
		\begin{itemize}
			\item \textbf{Chatbot integrado}: Incluye distintos niveles de chatbot (básico, avanzado y premium) para mejorar la interacción con los usuarios y proporcionar información automatizada.
			\item \textbf{Sistema de posicionamiento jerárquico}: Prioriza propiedades según el plan adquirido (estándar, prioritario, destacado).
			\item \textbf{Membresías escalonadas}: Ofrece planes Freemium y Premium con diferentes beneficios y limitaciones.
			\item \textbf{Notificaciones automáticas}: El sistema envía notificaciones relevantes dentro de la aplicación y por medio de correo electrónico.
		\end{itemize}