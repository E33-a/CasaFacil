\section{Estrategia de Escalabilidad}
	\noindent La arquitectura de CasaFácil combina escalabilidad horizontal como eje principal, vertical para casos específicos, y un enfoque híbrido para balancear rendimiento y costos.
	
	\subsection*{Escalabilidad Horizontal (Scale-Out)}
		\noindent \textbf{Justificación:} Responde a picos de carga impredecibles (por ejemplo, inicio de semestre). Mejora rendimiento al distribuir la carga. \\
		
		\textbf{Características que lo requieren:} 
		\begin{itemize}
			\item Crecimiento elástico ante demanda variable.
			\item Tolerancia a fallos (elimina puntos únicos de fallo).
			\item Planes premium generan consultas y operaciones simultáneas.
		\end{itemize}
		
		\textbf{Componentes escalados horizontalmente:}
		\begin{itemize}
			\item \textbf{Microservicios:}
			\begin{itemize}
				\item Autenticación con JWT/OAuth detrás de API Gateway.
				\item Búsqueda con Elasticsearch y geolocalización.
				\item Mensajería usando Redis Pub/Sub y WebSockets.
			\end{itemize}
			\item \textbf{Bases de datos:}
			\begin{itemize}
				\item PostgreSQL con réplicas de lectura.
				\item MongoDB con sharding por colección.
			\end{itemize}
			\item \textbf{Alta disponibilidad:}
			\begin{itemize}
				\item Kubernetes con múltiples zonas de disponibilidad.
				\item PodDisruptionBudgets y Service Discovery dinámico.
			\end{itemize}
		\end{itemize}
	
	\subsection*{Escalabilidad Vertical (Scale-Up)}
		\noindent \textbf{Justificación:} Se usa en componentes con estado o requerimientos de consistencia fuerte. \\
		
		\textbf{Características que lo requieren:}
		\begin{itemize}
			\item Bases de datos maestras (transacciones ACID).
			\item Servicios intensivos en RAM o IOPS.
		\end{itemize}
		
		\textbf{Componentes escalados verticalmente:}
		\begin{itemize}
			\item PostgreSQL maestro (más CPU/RAM).
			\item Redis con mayor capacidad de memoria.
			\item Kafka con brokers optimizados para eventos.
		\end{itemize}
	
	\subsection*{Enfoque Híbrido (Hybrid Scaling)}
		\noindent \textbf{Justificación:} Combina lo mejor de ambos modelos.
		
		\begin{itemize}
			\item \textbf{API Gateway:} Horizontal (balanceo de tráfico).
			\item \textbf{PostgreSQL:} Vertical + horizontal (maestro y réplicas).
			\item \textbf{Servicio de pagos:} Horizontal (procesamiento paralelo).
		\end{itemize}
		
		\textbf{Ventajas:}
		\begin{itemize}
			\item Flexibilidad ante demanda.
			\item Costos optimizados.
			\item Alta disponibilidad garantizada.
		\end{itemize}
	
	\subsection*{Alta Disponibilidad (99.9\%)}
	\begin{itemize}
		\item \textbf{Multi-AZ Deployment:} Microservicios y DBs en múltiples zonas.
		\item \textbf{Replicación:} PostgreSQL síncrona, MongoDB asíncrona.
		\item \textbf{Health Checks:} Kubernetes y ELB supervisan salud del sistema.
		\item \textbf{Backups:} Snapshots automáticos (RDS, MongoDB Atlas).
	\end{itemize}