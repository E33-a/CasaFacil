\section{Plan de Pruebas}
	\subsection*{Objetivos del Plan de Pruebas}
	\begin{itemize}
		\item Verificar funcionalidad de microservicios individuales.
		\item Detectar errores en etapas tempranas.
		\item Validar integridad de datos.
		\item Comprobar comportamiento ante entradas válidas e inválidas.
		\item Asegurar correcto funcionamiento aislado de cada módulo.
	\end{itemize}
	
	\subsection*{Microprocesos Críticos a Validar}
	\begin{itemize}
		\item \textbf{Cuentas:} Registro e inicio de sesión (JWT, validación).
		\item \textbf{Propiedades:} Creación, edición y moderación.
		\item \textbf{Reservas:} Disponibilidad, confirmación, rechazo.
		\item \textbf{Pagos:} Integración con MercadoPago, comprobantes.
		\item \textbf{Mensajería:} Envío y recepción, validación de longitud.
		\item \textbf{Mapas:} Búsqueda por cercanía, fallos de API externa.
	\end{itemize}
	
	\subsection*{Herramientas a Utilizar}
	\begin{itemize}
		\item \textbf{Pytest:} Pruebas unitarias.
		\item \textbf{Django TestCase:} Validación de vistas y modelos.
		\item \textbf{SQLite/PostgreSQL:} Bases de prueba.
		\item \textbf{Mock/Patch:} Simulación de servicios externos.
		\item \textbf{Factory Boy:} Generación de datos.
		\item \textbf{Coverage.py:} Medición de cobertura.
	\end{itemize}
	
	\subsection*{Estructura del Plan de Pruebas}
		\subsubsection*{Servicio de Cuentas}
			\begin{itemize}
				\item Registro válido → usuario creado y correo enviado.
				\item Registro duplicado → error de validación.
				\item Login exitoso → token JWT generado.
				\item Login fallido → error 401.
			\end{itemize}
		
		\subsubsection*{Servicio de Propiedades}
			\begin{itemize}
				\item Crear → pendiente de aprobación.
				\item Editar → reflejo de cambios.
				\item Moderación → cambia a estado visible.
			\end{itemize}
		
		\subsubsection*{Servicio de Reservas}
			\begin{itemize}
				\item Reserva exitosa → disponibilidad actualizada.
				\item Reserva fallida → acción bloqueada.
			\end{itemize}
		
		\subsubsection*{Servicio de Pagos}
			\begin{itemize}
				\item Pago correcto → HTTP 200, comprobante generado.
				\item Reembolso válido → estado actualizado.
			\end{itemize}
		
		\subsubsection*{Servicio de Mensajería}
			\begin{itemize}
				\item Mensaje válido → reenviado vía WebSocket.
				\item Exceso de caracteres → envío bloqueado.
			\end{itemize}
		
		\subsubsection*{Servicio de Mapas}
			\begin{itemize}
				\item Coordenadas válidas → propiedades cercanas.
				\item Error API → excepción manejada con mensaje.
			\end{itemize}
	
	\subsection*{Cobertura y Técnicas}
		\begin{itemize}
			\item \textbf{Cobertura esperada:} Mínimo 90\% en microservicios críticos.
			\item \textbf{Técnicas:}
			\begin{itemize}
				\item Particiones equivalentes.
				\item Valores límite.
				\item Simulación de fallos externos.
				\item Verificación antes/después de operación.
				\item Inyección de dependencias con mocks.
			\end{itemize}
		\end{itemize}
	
	\subsection*{Gestión y Ejecución}
		\begin{itemize}
			\item Automatización con Pytest.
			\item Ejecución en CI/CD (GitHub Actions o GitLab CI).
			\item Reportes con \texttt{coverage.py}.
			\item Alertas automáticas ante fallos.
		\end{itemize}
