\section{Roles del sistema}
	\noindent El sistema Casa Fácil se divide en varios roles para garantizar la seguridad, el control de acceso y la funcionalidad adecuada según el tipo de usuario. Los roles son los siguientes:
	
	\subsection*{Usuario no autenticado (Guest)}
		\begin{itemize}
			\item Puede explorar propiedades sin registrarse.
			\item No puede interactuar con otros usuarios ni realizar acciones críticas (como reservas o publicaciones).
			\item Solo tiene acceso a información pública y visualización de propiedades.
		\end{itemize}
	
	\subsection*{Usuario autenticado (User)}
		\begin{itemize}
			\item Puede buscar y visualizar propiedades según filtros definidos (ubicación, precio, servicios).
			\item Puede agregar propiedades a favoritos.
			\item Puede enviar mensajes a los propietarios para solicitar más información.
			\item Puede realizar reservas de inmuebles, siempre que estén disponibles.
			\item Puede ver su historial de reservas y transacciones.
			\item Puede calificar y dejar reseñas sobre propiedades al finalizar una reserva.
			\item Puede agendar visitas presenciales o virtuales a propiedades.
			\item Puede recibir notificaciones internas y por correo electrónico.
		\end{itemize}
	
	\subsection*{Propietario (Owner)}
		\begin{itemize}
			\item Puede crear y gestionar sus propiedades, incluyendo edición y eliminación.
			\item Puede subir imágenes, descripciones y especificaciones de cada propiedad.
			\item Puede programar la fecha de publicación de sus propiedades.
			\item Puede responder a mensajes de usuarios interesados en sus propiedades.
			\item Puede ver el estado de disponibilidad de sus propiedades.
			\item El número de propiedades que puede publicar está limitado según el paquete adquirido:
			\begin{itemize}
				\item Paquete Básico: 5 propiedades activas.
				\item Paquete Estándar: 10 propiedades activas.
				\item Paquete Premium: 15 propiedades activas.
			\end{itemize}
			\item Puede ver el historial de reservas y contactos relacionados con sus propiedades.
		\end{itemize}
	
	\subsection*{Administrador (Admin)}
		\begin{itemize}
			\item Tiene permisos completos para gestionar usuarios, propiedades, reservas y transacciones.
			\item Puede realizar operaciones CRUD (Crear, Leer, Actualizar, Eliminar) sobre usuarios, propiedades y transacciones.
			\item Puede gestionar permisos y roles dentro del sistema.
			\item Puede descargar información en diferentes formatos (CSV, PDF, Excel).
			\item Puede revisar y moderar reportes de usuarios o propiedades.
			\item Puede generar estadísticas y métricas del sistema.
			\item El rol de administrador puede ser segmentado en niveles de permiso para escalabilidad:
			\begin{itemize}
				\item \textbf{Administrador General}: Acceso total a todas las funciones del sistema.
				\item \textbf{Administrador de Contenido}: Solo gestiona propiedades, usuarios y reportes.
				\item \textbf{Administrador de Seguridad}: Gestionar usuarios, roles y permisos.
				\item \textbf{Administrador de Reportes}: Solo visualiza y genera informes.
			\end{itemize}
	\end{itemize}